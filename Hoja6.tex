\documentclass{article}
\usepackage[spanish,activeacute]{babel}
\usepackage{graphicx}
\usepackage{lmodern}
\usepackage[T1]{fontenc}

\author{Edwin L\'opez}
\title {\textbf{Hoja de trabajo No. 6}}
\date {26 de marzo del 2018}

\begin{document}

%Introducción al documento
\maketitle


%Seccion de ejemplos para los "Que hacer"
\section{"Primer Ejercicio"}
Para poder realizar el ejercicio se necesita verificar que algun n\'umero no sea cero, esto para poder utilizar una funci\'on llamada esCero (Numero a)

\begin{itemize}

\item Funci\'on Sumar(Numero a,Numero b) := si esCero(a) retorno b, si esCero(b) retorno a, Numero c es el predeccesor del numero a de lo contrario retorno un nuevo numero que es igual a Suma(c,b)
- en terminos de la funcion MayorQue() se declara la funcion Multiplicar

\item Funci\'on Multiplicar(Numero a, Numero b) := Numero nAcumulado := nuevo numero, si esCero(a) o esCero(b) retornar cero, de lo contrario retornar si MayorQue(b > 1) := Suma(-a,Multiplica(a,b.Predecesor)) - se necesita de una funcion de saver si los numeros son iguales SonIguales(Numero a,Numero b) //una funcion que retorna un valor booleano

\end{itemize}


\end{document} %finaliza todo el documento
