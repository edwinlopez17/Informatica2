\documentclass{article}
\usepackage[spanish,activeacute]{babel}
\usepackage{graphicx}
\usepackage{lmodern}
\usepackage[T1]{fontenc}

\author{Edwin L\'opez}
\title {\textbf{Hoja de trabajo No. 1}}
\date {24 de Enero del 2018}

\begin{document}

%Introducción al documento
\maketitle


%Seccion de ejemplos para los "Que hacer"
\section{"Que Hacer"}

\begin{itemize}

\item \textbf{Tiempo:} \textit{String, para saber el tiempo en que se realiza la tarea.}
\item \textbf{Fecha:} \textit{ String, para determinar d\'ia, mes, a\~no del "Que hacer".}
\item \textbf{Lugar:} \textit{String, para saber en donde se realiza.}
\item \textbf{Integrantes:} \textit{Int, para saber con cuantos integrantes lo realiza.}
\item \textbf{Motivo:} \textit{String, para saber por que lo realiza.}

\end{itemize}

%sección de los "Que haceres"
\section{"Que Haceres"}
%Inicia los items con estilo de punto, con estilo negrita y cursiva
\begin{itemize}

\item \textbf{Informar:} \textit{Informa el estado de la actividad.}
\item \textbf{Crear:} \textit{Crea una nueva actividad.}
\item \textbf{Buscar:} \textit{ Busca una nueva acci\'on para realizar.}
\item \textbf{Prioridad:} \textit{Da importancia a ciertas caracter'isticas.}
\item \textbf{Mezclar:} \textit{Mezcla las diferentes tareas a realizar.}

\end{itemize}


\end{document} %finaliza todo el documento
